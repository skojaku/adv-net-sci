\documentclass[a4paper, 14pt]{extarticle}
\usepackage[top=1in, bottom=1in, left=1in, right=1in]{geometry}
\usepackage{amsmath}
\usepackage{amssymb}
\usepackage{graphicx}
\usepackage{fontspec}
\usepackage{hyperref}
\usepackage{tikz}
\usepackage{fontspec}
%\usetikzlibrary{decorations.pathmorphing}
\usetikzlibrary{calc,decorations,patterns,arrows,decorations.pathmorphing,positioning}
\definecolor{pltblue}{HTML}{1F77B4}
\tikzset{every picture/.style={/utils/exec={\fontspec{Pretty Neat}}}}
\setmainfont{Pretty Neat}


\makeatletter
\pgfset{
  /pgf/decoration/randomness/.initial=2,
  /pgf/decoration/wavelength/.initial=100
}
\pgfdeclaredecoration{sketch}{init}{
  \state{init}[width=0pt,next state=draw,persistent precomputation={
    \pgfmathsetmacro\pgf@lib@dec@sketch@t0
  }]{}
  \state{draw}[width=\pgfdecorationsegmentlength,
  auto corner on length=\pgfdecorationsegmentlength,
  persistent precomputation={
    \pgfmathsetmacro\pgf@lib@dec@sketch@t{mod(\pgf@lib@dec@sketch@t+pow(\pgfkeysvalueof{/pgf/decoration/randomness},rand),\pgfkeysvalueof{/pgf/decoration/wavelength})}
  }]{
    \pgfmathparse{sin(2*\pgf@lib@dec@sketch@t*pi/\pgfkeysvalueof{/pgf/decoration/wavelength} r)}
    \pgfpathlineto{\pgfqpoint{\pgfdecorationsegmentlength}{\pgfmathresult\pgfdecorationsegmentamplitude}}
  }
  \state{final}{}
}
\tikzset{xkcd/.style={decorate,decoration={sketch,segment length=0.5pt,amplitude=0.5pt}}}
\makeatother

\usepackage{etoolbox}
\AtBeginEnvironment{tabular}{\fontspec{Pretty Neat}}

\setlength{\parindent}{0pt}
\setlength{\parskip}{0.5em}
\usepackage{fancyhdr}
\usepackage{geometry}
\usepackage{adjustbox}
\usepackage{titling}

\usepackage{amsmath}
\usepackage{amssymb}
\usepackage{graphicx}
\usepackage{hyperref}
\usepackage{tikz}
\usepackage{fontspec}
\usetikzlibrary{calc,decorations,patterns,arrows,decorations.pathmorphing}
\definecolor{pltblue}{HTML}{1F77B4}

\usepackage{amsmath}
\usepackage{tikz}
\usepackage{enumitem}

\title{Random Walk Characteristics Worksheet}
\author{Network Science Course}
\date{}

\begin{document}

%\maketitle
\section{Understanding Outer Products}

\begin{enumerate}
    \item Consider two vectors: $\mathbf{a} = [1, 2, 3]$ and $\mathbf{b} = [1,3,1]$. Calculate their outer product $\mathbf{a} \otimes \mathbf{b}$.

    \vspace{3cm}
    \item Without exact calculations, sketch the resulting matrix from the outer product $\mathbf{u} \otimes \mathbf{v}$ of the following pairs of vectors.
    \begin{enumerate}
        \item $\mathbf{u} = [5,4,3,2,1]$ and $\mathbf{v} = [5,4,3,2,1]$
        \item $\mathbf{u} = [1,2,4,0,0]$ and $\mathbf{v} = [3, 1, 2, 0, 0]$
        \item $\mathbf{u} = [1,1,1,-1,-1]$ and $\mathbf{v} = [1,1,1,-1,-1]$
    \end{enumerate}
    Use shading to represent the relative values in the matrix (darker shades for larger values).
    \begin{center}
    \begin{tabular}{ccc}
        (a) & (b) & (c) \\
        \begin{tabular}{|c|c|c|c|c|c|}
        \hline
        & 1 & 2 & 3 & 4 & 5 \\
        \hline
        1 & & & & & \\
        \hline
        2 & & & & & \\
        \hline
        3 & & & & & \\
        \hline
        4 & & & & & \\
        \hline
        5 & & & & & \\
        \hline
        \end{tabular}
        &
        \begin{tabular}{|c|c|c|c|c|c|}
        \hline
        & 1 & 2 & 3 & 4 & 5 \\
        \hline
        1 & & & & & \\
        \hline
        2 & & & & & \\
        \hline
        3 & & & & & \\
        \hline
        4 & & & & & \\
        \hline
        5 & & & & & \\
        \hline
        \end{tabular}
        &
        \begin{tabular}{|c|c|c|c|c|c|}
        \hline
        & 1 & 2 & 3 & 4 & 5 \\
        \hline
        1 & & & & & \\
        \hline
        2 & & & & & \\
        \hline
        3 & & & & & \\
        \hline
        4 & & & & & \\
        \hline
        5 & & & & & \\
        \hline
        \end{tabular}
    \end{tabular}
    \end{center}
\end{enumerate}

\begin{enumerate}[resume]
    \item Look at the matrices you've sketched. If these matrices represented networks, what kind of network structures might each of them represent?
\end{enumerate}

\section{Decomposing Matrices}

\begin{enumerate}[resume]
    \item Consider the following matrix representing a small network:

    \[
    \begin{bmatrix}
    3 & 2 & 1 & 0 & 1 \\
    2 & 3 & 0 & 1 & 0 \\
    1 & 0 & 3 & 2 & 1 \\
    0 & 1 & 2 & 4 & 2 \\
    1 & 0 & 1 & 2 & 3
    \end{bmatrix}
    \]

    Try to ``decompose'' this matrix into the outer product of two vectors with minimal error. Sketch the vectors and use shading to represent their values.
    \vspace{3cm}
    \item Now consider this matrix:

    \[
    \begin{bmatrix}
    8 & 4 & 2 & 1 & 0 \\
    4 & 4 & 2 & 0 & 0 \\
    2 & 2 & 2 & 1 & 0 \\
    1 & 0 & 1 & 1 & 1 \\
    0 & 0 & 0 & 1 & 1
    \end{bmatrix}
    \]

    Can you decompose this into a sum of two outer products? Sketch the vectors for each outer product and use shading.
    \vspace{3cm}
    \item For the matrix in question 7, if you had to keep only one of the two outer products, which would you choose and why? What information about the network would be preserved?
    \vspace{3cm}
\end{enumerate}

\end{document}